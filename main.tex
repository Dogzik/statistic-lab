\documentclass{article}
\usepackage[utf8]{inputenc}
\usepackage{listings}
\lstset{
    language=Octave,
    frame=single,
    xleftmargin=.1\textwidth, xrightmargin=.1\textwidth
}
\usepackage[T2A]{fontenc}
\usepackage[utf8]{inputenc}
\usepackage[russian]{babel}
\usepackage[left=2cm,right=2cm,top=2cm,bottom=2.1cm,bindingoffset=0cm]{geometry}
\usepackage{mathtools}

\title{Теория Вероятности и Математическая Статистика}
\author{Довжик Лев M3339}
\date{Май 2019}

\begin{document}
\maketitle
\begin{center}
    Метод Монте-Карло \\
    Вариант №8
\end{center}

\section{Оценка объёма}
    \subsection{Задание}
        Методом Монте-Карло оценить объем части тела \{$F(\tilde x) \leq c$\}, заключенной в $k$-мерном кубе с ребром $[0, 1]$. 
        Функция имеет вид $F(\tilde x) = f(x_1) + f(x_2) + ... + f(x_k)$.
        Для выбранной надежности $\gamma \geq 0.95$ указать асимптотическую точность оценки и построить асимптотический доверительный интервал для истинного значения объёма. \\
        Используя объём выборки $n = 10^4$ и $n = 10^6$, оценить скорость сходимости и показать, что доверительные интервалы пересекаются.
    \subsection{Входные данные}
        \begin{itemize}
            \item Функция имеет вид $f(x) = \ln(9x + 1)$
            \item Куб размерностью $k = 6$
            \item Параметр $c = 8.61$ 
        \end{itemize}
    \subsection{Исходный код программы}
        \begin{minipage}{\linewidth}
            \lstinputlisting{figureVolume.m}
        \end{minipage}
    \subsection{Выходные данные}
    \begin{center}
\begin{tabular}{c}
        \begin{minipage}{\linewidth}
            \lstinputlisting{figureOutput.txt}
        \end{minipage}
        \end{tabular}
        \end{center}
    \subsection{Вывод}
        Доверительный интервал при $n = 10^6$ содержится в интервале при $n = 10^4$.\\
        При увеличении числа итераций в $100$ раз ширина доверительного интервала уменьшилось в $10$ раз.
\section{Оценка Интеграла}
    \subsection{Задание}
    Построить оценку интегралов (представить интеграл как математическое ожидание функции,
    зависящей от случайной величины с известной плотностью) и для выбранной надежности $\gamma \geq 0.95$ указать
    асимптотическую точность оценки и построить асимптотический доверительный интервал для истинного
    значения интеграла. 
    \subsection{Интеграл 1}
        \subsubsection{Входные данные}
            \begin{itemize}
                \item Интеграл имеет вид ${\displaystyle \int\limits_{-\infty}^{\infty} \left(x - 2\right)^3 exp \left( \dfrac{-(x - 1)^2}{3} \right) dx}$
            \end{itemize}
        \subsubsection{Исходный код программы}
        \begin{minipage}{\linewidth}
            \lstinputlisting{integralValue1.m}
        \end{minipage}
        \subsubsection{Выходные данные}
        \begin{minipage}{\linewidth}
            \lstinputlisting{integralOutput1.txt}
        \end{minipage}
        \subsubsection{Вывод}
        Истинное значение интеграла содержится в доверительном интервале при $n = 10^4$ и $n = 10^6$. Значение, полученное методом Монте-Карло отличается от значения, полученного методом $quad$, на $7.2 \cdot 10^{-3}$.\\
        При увеличении числа итераций в $100$ раз, ширина доверительного интервала уменьшилось в $10$ раз.
    \subsection{Интеграл 2}
        \subsubsection{Входные данные}
            \begin{itemize}
                \item Интеграл имеет вид ${\displaystyle \int\limits_{-1}^{2} \sqrt{1 + x} \cos x dx}$
            \end{itemize}
        \subsubsection{Исходный код программы}
        \begin{minipage}{\linewidth}
            \lstinputlisting{integralValue2.m}
        \end{minipage}
        \subsubsection{Выходные данные}
        \begin{minipage}{\linewidth}
            \lstinputlisting{integralOutput2.txt}
        \end{minipage}
        \subsubsection{Вывод}
        Истинное значение интеграла содержится в доверительном интервале при $n = 10^4$ и $n = 10^6$. Значение, полученное методом Монте-Карло отличается от значения, полученного методом $quad$, на $2.8 \cdot 10^{-4}$.\\
        При увеличении числа итераций в $100$ раз, ширина доверительного интервала уменьшилось в $10$ раз.
\end{document}
